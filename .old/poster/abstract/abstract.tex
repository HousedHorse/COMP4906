% Options for packages loaded elsewhere
\PassOptionsToPackage{unicode}{hyperref}
\PassOptionsToPackage{hyphens}{url}
\PassOptionsToPackage{dvipsnames,svgnames*,x11names*}{xcolor}
%
\documentclass[
  12pt]{findlay}
\usepackage{lmodern}
\usepackage{amssymb,amsmath}
\usepackage{ifxetex,ifluatex}
\ifnum 0\ifxetex 1\fi\ifluatex 1\fi=0 % if pdftex
  \usepackage[T1]{fontenc}
  \usepackage[utf8]{inputenc}
  \usepackage{textcomp} % provide euro and other symbols
\else % if luatex or xetex
  \usepackage{unicode-math}
  \defaultfontfeatures{Scale=MatchLowercase}
  \defaultfontfeatures[\rmfamily]{Ligatures=TeX,Scale=1}
\fi
% Use upquote if available, for straight quotes in verbatim environments
\IfFileExists{upquote.sty}{\usepackage{upquote}}{}
\IfFileExists{microtype.sty}{% use microtype if available
  \usepackage[]{microtype}
  \UseMicrotypeSet[protrusion]{basicmath} % disable protrusion for tt fonts
}{}
\makeatletter
\@ifundefined{KOMAClassName}{% if non-KOMA class
  \IfFileExists{parskip.sty}{%
    \usepackage{parskip}
  }{% else
    \setlength{\parindent}{0pt}
    \setlength{\parskip}{6pt plus 2pt minus 1pt}}
}{% if KOMA class
  \KOMAoptions{parskip=half}}
\makeatother
\usepackage{xcolor}
\IfFileExists{xurl.sty}{\usepackage{xurl}}{} % add URL line breaks if available
\IfFileExists{bookmark.sty}{\usepackage{bookmark}}{\usepackage{hyperref}}
\hypersetup{
  colorlinks=true,
  linkcolor=black,
  filecolor=Maroon,
  citecolor=Green,
  urlcolor=blue,
  pdfcreator={LaTeX via pandoc}}
\urlstyle{same} % disable monospaced font for URLs
\usepackage{listings}
\newcommand{\passthrough}[1]{#1}
\lstset{defaultdialect=[5.3]Lua}
\lstset{defaultdialect=[x86masm]Assembler}
\setlength{\emergencystretch}{3em} % prevent overfull lines
\providecommand{\tightlist}{%
  \setlength{\itemsep}{0pt}\setlength{\parskip}{0pt}}
\setcounter{secnumdepth}{5}
\makeatletter
\def\@maketitle{%
  \begin{center}%
  \let \footnote \thanks
    {\large\scshape\bfseries \@title \par}%
    \vskip 1em
    {\large
      \begin{tabular}[t]{c}%
        \@author
      \end{tabular}}%
    \vskip 1em%
    {\large \@date}
  \end{center}%
  \par}
\makeatother
\usepackage[]{biblatex}

\title{EXTENDED BERKELEY PACKET FILTER FOR INTRUSION DETECTION
IMPLEMENTATIONS}
\author{William Findlay\\
\small School of Computer Science, Carleton University}
\date{}

\begin{document}
\maketitle

\thispagestyle{empty}

System introspection is becoming an increasingly attractive option for
maintaining operating system stability and security. This is primarily
due to the many recent advances in system introspection technology; in
particular, the 2013 introduction of \emph{Extended Berkeley Packet
Filter} (\emph{eBPF}) into the Linux Kernel along with the recent
development of more usable interfaces such as the \emph{BPF Compiler
Collection} (\emph{bcc}) has resulted in a highly compelling,
performant, and (perhaps most importantly) safe subsystem for both
kernel and userland instrumentation.

The scope, safety, and performance of eBPF system introspection has
potentially powerful applications in the domain of computer security. In
order to demonstrate this, we present \emph{ebpH}, an eBPF
implementation of Somayaji's \emph{Process Homeostasis} (\emph{pH}).
ebpH is an intrusion detection system (IDS) that uses eBPF programs to
instrument system calls and establish normal behavior for processes,
building a profile for each executable on the system; subsequently, ebpH
can warn the user when it detects process behavior that violates the
established profiles.

This poster will discuss the design and implementation of ebpH along
with the technical challenges which occurred along the way. We will
present experimental data and performance benchmarks that demonstrate
ebpH's ability to monitor process behavior with minimal overhead.
Finally, we conclude with a discussion on the merits of eBPF IDS
implementations and potential avenues for future work therein.

\printbibliography

\end{document}
