% Options for packages loaded elsewhere
\PassOptionsToPackage{hyphens}{url}
\def\UrlBreaks{\do\/\do-\do.\do=\do_\do?\do\&\do\%\do\a\do\b\do\c\do\d\do\e\do\f\do\g\do\h\do\i\do\j\do\k\do\l\do\m\do\n\do\o\do\p\do\q\do\r\do\s\do\t\do\u\do\v\do\w\do\x\do\y\do\z\do\A\do\B\do\C\do\D\do\E\do\F\do\G\do\H\do\I\do\J\do\K\do\L\do\M\do\N\do\O\do\P\do\Q\do\R\do\S\do\T\do\U\do\V\do\W\do\X\do\Y\do\Z\do\0\do\1\do\2\do\3\do\4\do\5\do\6\do\7\do\8\do\9}
\documentclass[
  12pt]{findlay}

% Pandoc Stuff
\makeatletter
\RequirePackage{listings}
\RequirePackage{graphicx}
\def\maxwidth{\ifdim\Gin@nat@width>\linewidth\linewidth\else\Gin@nat@width\fi}
\def\maxheight{\ifdim\Gin@nat@height>\textheight\textheight\else\Gin@nat@height\fi}
% Scale images if necessary, so that they will not overflow the page
% margins by default, and it is still possible to overwrite the defaults
% using explicit options in \includegraphics[width, height, ...]{}
\setkeys{Gin}{width=\maxwidth,height=\maxheight,keepaspectratio}
% Set default figure placement to htb
\def\fps@figure{htb!}
\newcommand{\passthrough}[1]{#1}
\providecommand{\tightlist}{\setlength{\itemsep}{0pt}\setlength{\parskip}{0pt}}
\makeatother




\title{Speaking Notes}
\author{William Findlay}
\date{\today}


\begin{document}

\newpage
\pagenumbering{arabic}
\setcounter{page}{1}

\hypertarget{slide-1}{%
\section{Slide 1}\label{slide-1}}

\begin{itemize}
\tightlist
\item
  Introduce myself
\item
  Here to talk about eBPF

  \begin{itemize}
  \tightlist
  \item
    all the crazy things you can do with it
  \item
    how it can be a crazy powerful tool for OS security
  \item
    demonstrate this using ebpH, a host-based anomaly detection system I
    built with eBPF
  \end{itemize}
\end{itemize}

\hypertarget{slide-2}{%
\section{Slide 2}\label{slide-2}}

\begin{itemize}
\tightlist
\item
  At a high level, eBPF is a way of injecting user-specified code into
  the kernel
\item
  With the idea being that this code runs in kernelspace and can be used
  to instrument essentially all system behavior

  \begin{itemize}
  \tightlist
  \item
    that means we can see all userspace and all kernelspace events, all
    at the same time
  \end{itemize}
\item
  The key advantage of eBPF over traditional methods such as LKMs is
  this idea of safety

  \begin{itemize}
  \tightlist
  \item
    BPF programs are guaranteed not to crash the kernel or damage a
    running system
  \item
    they have a verifier for this, which I will explain shortly
  \end{itemize}
\end{itemize}

\hypertarget{slide-3}{%
\section{Slide 3}\label{slide-3}}

\begin{itemize}
\tightlist
\item
  Before I talk about security-specific use cases, I want to touch on
  how pervasive this technology already is in industry
\item
  Lots of big data companies are already using eBPF for performance
  monitoring, debugging, optimization

  \begin{itemize}
  \tightlist
  \item
    this is happening in production and at scale
  \item
    companies like Netflix, Facebook, Google, many more
  \item
    creator of eBPF actually works at Facebook
  \end{itemize}
\item
  There is also an established set of tools that cover most basic use
  cases

  \begin{itemize}
  \tightlist
  \item
    bcc (sponsored by the Linux foundation) has over 100 of them
  \item
    monitor everything from cache hits/misses to TCP connection requests
    to setuid system calls
  \end{itemize}
\item
  There's also a really important example of eBPF being used for
  security in production

  \begin{itemize}
  \tightlist
  \item
    Cloudflare's DDoS mitigation stack is now largely based on eBPF
  \end{itemize}
\end{itemize}

\hypertarget{slide-4}{%
\section{Slide 4}\label{slide-4}}

\begin{itemize}
\tightlist
\item
  Now that you know a bit about the kinds of things you can do with
  eBPF, we need to talk about how this relates to security
\item
  I think this really starts with the notion that a lot of security is
  about what you can see

  \begin{itemize}
  \tightlist
  \item
    if an attacker can break into a system and we don't know about it
    until it's too late, they've gotten away with it
  \item
    remember, eBPF lets us see everything about our system
  \item
    that means every system call, every login, every scheduler
    operation, every packet, all at the same time
  \item
    and it can do this with crazy low overhead
  \end{itemize}
\item
  Before eBPF, this kind of system introspection came at a cost

  \begin{itemize}
  \tightlist
  \item
    generally you have a trade-off between speed, scope, or production
    safety
  \item
    library call interposition, kernel modification, ptrace is just
    terrible (REALLY slow, up to 44,200\%)
  \end{itemize}
\item
  eBPF can do everything without having to deal with this speed / scope
  / safety trade-off
\end{itemize}

\hypertarget{slide-5}{%
\section{Slide 5}\label{slide-5}}

\begin{itemize}
\tightlist
\item
  Up until now I've just been handwaving and making all these magical
  claims about eBPF, so here is how it actually works
\item
  We start in userspace where we first need to generate some BPF
  bytecode to send off to the kernel

  \begin{itemize}
  \tightlist
  \item
    originally this was done by hand, but now we have fancy LLVM
    backends that can convert C into BPF instructions
  \end{itemize}
\item
  Once we have our bytecode, we send it off to the kernel using the BPF
  system call, and this system call immediately traps to the eBPF
  verifier
\item
  The verifier checks our code for safety and will either immediately
  reject our code or accept our code, at which point it moves onto a JIT
  compiler which converts it in real time to machine instructions
\item
  Once the BPF program is running, there are a lot of different things
  that we can use it to instrument

  \begin{itemize}
  \tightlist
  \item
    userspace functions
  \item
    kernel functions / data structures
  \item
    hardware performance counters
  \item
    networking stuff
  \end{itemize}
\item
  Everything is event based, so this BPF program is being run every time
  a specific event occurs
\item
  All BPF programs can store data in maps

  \begin{itemize}
  \tightlist
  \item
    BPF programs can use maps to interact with each other
  \item
    And maps can be used to interact with userspace (via the BPF system
    call)
  \item
    The important thing to understand about these maps is that we can
    control the amount of context switches we need to do, which is a lot
    of where BPF's performance comes from, over something like Dtrace
  \end{itemize}
\end{itemize}

\hypertarget{slide-6}{%
\section{Slide 6}\label{slide-6}}

\begin{itemize}
\tightlist
\item
  The verifier's job is to make sure that BPF programs do not crash the
  kernel
\item
  It can be considered part of the kernel's reference monitor
\item
  10,000 lines of C code, pretty complex
\item
  BPF system call traps to verifier on every PROG\_LOAD
\item
  How can you guarantee safety?

  \begin{itemize}
  \tightlist
  \item
    limitations
  \item
    simulation
  \item
    static analysis
  \end{itemize}
\item
  512 byte stack space
\item
  No unbounded loops
\item
  Hard limit of 4 million BPF instructions per program
\item
  No buffer access with unbounded induction variables
\end{itemize}

\hypertarget{slide-7}{%
\section{Slide 7}\label{slide-7}}

\begin{itemize}
\tightlist
\item
  Even with all these limitations, BPF programs can still be quite
  complex
\item
  The figure on the left is an instruction flow graph of one of ebpH's
  BPF programs
\item
  Generated with bpftool and graphviz osage
\item
  Just over 1500 BPF instructions, JIT compiled to just over 1900
  machine instructions
\item
  A lot of additional complexity comes from maps + tail calls
\end{itemize}

\hypertarget{slide-8}{%
\section{Slide 8}\label{slide-8}}

\begin{itemize}
\tightlist
\item
  Before I talk about my anomaly detection system, ebpH, I'll talk a bit
  about what came before it
\item
  pH or Process Homeostasis was an early anomaly detection by Anil
\item
  The idea was to instrument system calls and build behavioral profiles

  \begin{itemize}
  \tightlist
  \item
    Delay anomalous system calls proportionally to recent anomalies
  \end{itemize}
\item
  Implementation had a few key problems

  \begin{itemize}
  \tightlist
  \item
    The stuff that pH was doing is basically impossible to do outside of
    patching the kernel directly
  \item
    Crazy modifications like patching the scheduler, writing new
    assembly code, etc.
  \item
    The key points here: not production safe, certainly not portable
  \end{itemize}
\end{itemize}

\hypertarget{slide-9}{%
\section{Slide 9}\label{slide-9}}

\begin{itemize}
\tightlist
\item
  ebpH is a portmanteau of Extended BPF and Process Homeostasis
\item
  The idea is taking this 20 year old technology and re-writing it using
  modern technology
\item
  Table below compares the original pH and the current version ebpH in
  some important categories
\item
  (NEXT SLIDE)
\end{itemize}

\hypertarget{slide-10}{%
\section{Slide 10}\label{slide-10}}

\begin{itemize}
\tightlist
\item
  Firstly, I want to point out that ebpH solves the problems of
  portability and production safety
\item
  Two innate properties of BPF

  \begin{itemize}
  \tightlist
  \item
    perfect forward compatibility (we generally only add features, don't
    take them away)
  \item
    verifier's safety guarantees
  \end{itemize}
\item
  (NEXT SLIDE)
\end{itemize}

\hypertarget{slide-11}{%
\section{Slide 11}\label{slide-11}}

\begin{itemize}
\tightlist
\item
  Note that pH actually has two properties that ebpH currently does not
  satisfy

  \begin{itemize}
  \tightlist
  \item
    Low memory overhead and a response component (the system call
    delays)
  \end{itemize}
\item
  Future work covers this, hopefully I'll be able to convince you that
  future versions of ebpH will be able to satisfy all of my design goals
\end{itemize}

\hypertarget{slide-12}{%
\section{Slide 12}\label{slide-12}}

\begin{itemize}
\tightlist
\item
  Basic idea of ebpH is the same as pH

  \begin{itemize}
  \tightlist
  \item
    Trace system calls
  \item
    Build profile of lookahead pairs
  \item
    Gather enough data to normalize a profile
  \item
    New lookahead pair entries in a normal profile are considered
    anomalies
  \end{itemize}
\item
  eBPF makes this safe

  \begin{itemize}
  \tightlist
  \item
    We are doing this instrumentation in a new way
  \item
    The key difference here is data collection
  \end{itemize}
\item
  Figure on the left shows an example lookahead pair from the ls binary

  \begin{itemize}
  \tightlist
  \item
    (read, close)
  \end{itemize}
\end{itemize}

\hypertarget{slide-13}{%
\section{Slide 13}\label{slide-13}}

\begin{itemize}
\tightlist
\item
  ebpH uses three BPF program types to collect data
\item
  most important is tracepoints

  \begin{itemize}
  \tightlist
  \item
    instrument system calls
  \item
    instrument the scheduler for process creation and profile
    association on execve
  \end{itemize}
\item
  kprobes for dynamic instrumentation of signal delivery

  \begin{itemize}
  \tightlist
  \item
    this allows ebpH to be signal-aware, which helps it deal with
    non-deterministic behavior from signal handlers
  \end{itemize}
\item
  uprobes

  \begin{itemize}
  \tightlist
  \item
    instrument userspace shared library called libebph
  \item
    allows userspace to issue more complex commands to the BPF programs
  \item
    on top of direct map access like we have seen before
  \end{itemize}
\end{itemize}

\hypertarget{slide-14}{%
\section{Slide 14}\label{slide-14}}

\begin{itemize}
\tightlist
\item
  Daemon manages the lifecycle of all of ebpH's BPF programs

  \begin{itemize}
  \tightlist
  \item
    loads the BPF programs when it is started
  \item
    BPF programs are invoked every time specific events occur (system
    calls, process creation, binary loading, signals, etc.)
  \item
    continually polls buffered events from the BPF programs
    (e.g.~anomalies, profile creation, etc.)
  \item
    it also has direct access to all of the BPF hashmaps
  \end{itemize}
\item
  Profiles are stored persistently on disk, log files are kept which
  describe all ebpH events in detail
\item
  Daemon exposes an API to other userspace programs through a UNIX
  stream socket

  \begin{itemize}
  \tightlist
  \item
    this socket is owned by root and is only readable and writeable by
    root
  \end{itemize}
\end{itemize}

\hypertarget{slide-15}{%
\section{Slide 15}\label{slide-15}}

\begin{itemize}
\tightlist
\item
  I have made a lot of claims so far about eBPF's performance, so let's
  see how ebpH compares with the original pH implementation
\item
  Three distinct benchmarks
\item
  lmbench suite to test

  \begin{itemize}
  \tightlist
  \item
    basic system call overhead
  \item
    process creation overhead
  \item
    IPC overhead (signals, UDS, pipes) (won't be covering this category)
  \end{itemize}
\item
  kernel compilation benchmarks

  \begin{itemize}
  \tightlist
  \item
    how does ebpH affect the overhead of real, complex tasks
  \end{itemize}
\item
  bpfbench

  \begin{itemize}
  \tightlist
  \item
    wrote my own ad-hoc benchmarking tool in eBPF
  \item
    uses BPF programs to instrument latency on system call execution
  \item
    advantage: monitor entire system (see most frequent system calls in
    practice and measure their overhead)
  \end{itemize}
\end{itemize}

\hypertarget{slide-16}{%
\section{Slide 16}\label{slide-16}}

\begin{itemize}
\tightlist
\item
  bpfbench

  \begin{itemize}
  \tightlist
  \item
    three distinct systems, three distinct workloads
  \end{itemize}
\item
  micro benchmarks

  \begin{itemize}
  \tightlist
  \item
    bronte only (besides micro benchmarks, bronte was idle)
  \end{itemize}
\end{itemize}

\hypertarget{slide-17}{%
\section{Slide 17}\label{slide-17}}

\begin{itemize}
\tightlist
\item
  getppid

  \begin{itemize}
  \tightlist
  \item
    null system call (almost no kernelspace runtime)
  \item
    essentially involves a namespace lookup + a quick data structure
    access
  \item
    this shows the worst case for performance overhead
  \item
    about 614\%, which seems like a lot, but drops off quickly
  \end{itemize}
\item
  stat(2)

  \begin{itemize}
  \tightlist
  \item
    more kernelspace runtime
  \item
    this has much better overhead, 65\%
  \end{itemize}
\item
  select(2) blocking system call, approaches best case

  \begin{itemize}
  \tightlist
  \item
    blocks until property is true on one or more open file descriptors
  \item
    overhead as high as 99\%, as low as only 2\%
  \end{itemize}
\end{itemize}

\hypertarget{slide-18}{%
\section{Slide 18}\label{slide-18}}

\begin{itemize}
\tightlist
\item
  process creation

  \begin{itemize}
  \tightlist
  \item
    three test cases, from least to most complex
  \end{itemize}
\item
  fork + exit creates a new process and exits immediately (this almost
  never happens in the wild)
\item
  fork + execve creates a new process and executes a simple hello world
  program (this is probably most common)
\item
  fork + /bin/sh -c same as fork + execve except uses the shell to
  execute the hello world program
\item
  highest overhead is only 10\%, which is acceptable in practice
\end{itemize}

\hypertarget{slide-19}{%
\section{Slide 19}\label{slide-19}}

\begin{itemize}
\tightlist
\item
  kernel compilation benchmark shows the most impressive result
\item
  the idea was to test a very CPU-intensive task
\item
  involves a lot of userspace time (which ebpH doesn't really affect),
  but still many system calls

  \begin{itemize}
  \tightlist
  \item
    tested how many -\textgreater{} 176 million
  \end{itemize}
\item
  ebpH imposes only 10\% kernelspace overhead
\item
  the total overhead is under 1\%
\item
  this actually shows ebpH outperforming the original pH (kernel
  implementation)
\end{itemize}

\hypertarget{slide-20}{%
\section{Slide 20}\label{slide-20}}

\begin{itemize}
\tightlist
\item
  bpfbench results
\item
  a lot of data, I will summarize for the sake of time
\item
  looked at top 20 system calls by count across three datasets
\item
  overhead from about 5\% to 150\%, acceptable in practice
\item
  non-idle systems had longer system calls as most frequent

  \begin{itemize}
  \tightlist
  \item
    which translates to lower overhead where it really matters
  \end{itemize}
\end{itemize}

\hypertarget{slide-21}{%
\section{Slide 21}\label{slide-21}}

\begin{itemize}
\tightlist
\item
  in summary:
\item
  ebpH can impose significant overhead on some system calls

  \begin{itemize}
  \tightlist
  \item
    but this isn't the whole story
  \item
    longer system calls means less overhead
  \item
    systems doing real work tend to invoke longer system calls more
    frequently
  \item
    system call overhead is not necessarily indicative of tangible
    performance impact
  \end{itemize}
\item
  ebpH does VERY well on real tasks

  \begin{itemize}
  \tightlist
  \item
    outperforms the original pH
  \item
    slowdown is imperceptible in practice
  \end{itemize}
\end{itemize}

\hypertarget{slide-22}{%
\section{Slide 22}\label{slide-22}}

\begin{itemize}
\tightlist
\item
  we have seen the eBPF can be very effective in data collection
\item
  there are some new additions to eBPF that can make it very effective
  at responding to attacks
\item
  in Linux 5.3, we got bpf\_signal

  \begin{itemize}
  \tightlist
  \item
    send signals to processes in real time from kernelspace
  \item
    if a signal is coming from the kernel, it happens instantaneously
  \end{itemize}
\item
  Linux 5.5, bpf\_signal\_thread

  \begin{itemize}
  \tightlist
  \item
    like bpf\_signal, but we now control which thread receives the
    signal
  \end{itemize}
\item
  Linux 4.16, bpf\_override\_return

  \begin{itemize}
  \tightlist
  \item
    targeted error injection on whitelisted kernel functions
  \item
    block system calls from completing successfully
  \end{itemize}
\end{itemize}

\hypertarget{slide-23}{%
\section{Slide 23}\label{slide-23}}

\begin{itemize}
\tightlist
\item
  now we have the tools to make ebpH respond to attacks
\item
  send SIGSTOP and SIGCONT for system call delays
\item
  use bpf\_override\_return to stop anomalous execves
\item
  recall table 1 from the beginning of the talk
\end{itemize}

\hypertarget{slide-24}{%
\section{Slide 24}\label{slide-24}}

\begin{itemize}
\tightlist
\item
  I also want to touch on ways that ebpH's memory overhead can be
  improved
\item
  currently, one big map for processes, one big map for profiles

  \begin{itemize}
  \tightlist
  \item
    this is way too granular
  \end{itemize}
\item
  original pH used a linked list to keep track of processes and profiles

  \begin{itemize}
  \tightlist
  \item
    dynamic allocation in the traditional sense is not possible in BPF,
    needs to be done through maps
  \end{itemize}
\item
  new BPF map types can help make things more fine-grained

  \begin{itemize}
  \tightlist
  \item
    LRU\_HASH enables us to use a smaller map that discards least
    recently used entries
  \item
    HASH\_OF\_MAPS nested maps for the sparse array of lookahead pairs,
    achieve something like dynamic allocation
  \end{itemize}
\end{itemize}

\hypertarget{slide-25}{%
\section{Slide 25}\label{slide-25}}

\begin{itemize}
\tightlist
\item
  what other types of security problems can we solve with BPF?
\item
  anomaly detection

  \begin{itemize}
  \tightlist
  \item
    no reason to stop at system calls
  \item
    ebpH only uses a very tiny subset of eBPF's functionality
  \item
    there are over 1,800 tracepoints alone in Linux 5.5, ebpH uses five
    of them
  \end{itemize}
\item
  DDoS mitigation and network intrusion detection

  \begin{itemize}
  \tightlist
  \item
    this is already being done
  \item
    Cloudflare, ntopng, etc.
  \end{itemize}
\item
  Increasing visibility of attacks and misuse

  \begin{itemize}
  \tightlist
  \item
    ebpH does a bit of this
  \item
    bcc tools are great at this
  \item
    capable

    \begin{itemize}
    \tightlist
    \item
      monitor every time a capability is used (CAP\_SYS\_ADMIN,
      CAP\_NET\_BIND\_SERVICE, etc.)
    \end{itemize}
  \item
    eperm

    \begin{itemize}
    \tightlist
    \item
      monitor every eperm error (useful for constructing policies)
    \end{itemize}
  \item
    setuids

    \begin{itemize}
    \tightlist
    \item
      monitor all calls to setuid
    \end{itemize}
  \item
    execsnoop

    \begin{itemize}
    \tightlist
    \item
      monitor every execve system call
    \end{itemize}
  \item
    many others
  \end{itemize}
\end{itemize}

\hypertarget{slide-26}{%
\section{Slide 26}\label{slide-26}}

\begin{itemize}
\tightlist
\item
  something I've been thinking about recently is using eBPF to enforce
  sandboxing rules

  \begin{itemize}
  \tightlist
  \item
    something like seccomp, but enforce rules externally, totally
    application transparent
  \item
    bpf\_signal could do this easily
  \end{itemize}
\item
  the main takeaway from this talk should be as follows:

  \begin{itemize}
  \tightlist
  \item
    name anything you want to trace
  \item
    eBPF can do it and it can do it safely and with excellent
    performance
  \end{itemize}
\item
  ebpH is just the beginning
\end{itemize}

\hypertarget{slide-27}{%
\section{Slide 27}\label{slide-27}}

\begin{itemize}
\tightlist
\item
  in conclusion
\item
  ebpH is as fast or faster than the original kernel implementation
\item
  it supports most of the original functionality, and future versions
  will support all of it
\item
  we can expand on ebpH to leverage more system data with eBPF

  \begin{itemize}
  \tightlist
  \item
    I envision a whole ecosystem of BPF programs, monitoring various
    aspects of system state, sharing information with each other
  \item
    take things beyond system call tracing and truly begin to emulate
    natural homeostatic mechanisms
  \end{itemize}
\item
  the future of eBPF in OS security is bright

  \begin{itemize}
  \tightlist
  \item
    we are going to see a lot more of this kind of thing
  \item
    we are just at the beginning, and eBPF gets better with every
    subsequent kernel update
  \end{itemize}
\end{itemize}

\hypertarget{slide-28}{%
\section{Slide 28}\label{slide-28}}

\begin{itemize}
\tightlist
\item
  any questions?
\end{itemize}

\clearpage

\printbibliography

\end{document}
